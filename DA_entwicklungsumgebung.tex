\section{Entwicklungsumgebung}
Im folgenden Abschnitt werden die verschiedenen Entwicklungsumgebungen und ihre Funktionalitäten näher beschrieben.

\subsection{JetBrains PhpStorm}
PhpStorm ist die IDE\footnote{Integrierte Entwicklungsumgebung} für die Programmiersprache PHP der Firma JetBrains. Die IDE punktet durch ihre Schnelligkeit und eine große Anzahl an Funktionen, die von Haus aus mitgeliefert werden, wie zum Beispiel Syntax-Highlighting, intelligente Strukturverbesserung des Quellcodes usw. Das Produkt wird zudem ständig aktualisiert und kann durch zahlreiche Erweiterungen noch weiter verbessert werden. (vgl. \cite{phpstorm})  

Für detailliertere Informationen siehe Hersteller-Webseite.\footnote{\url{https://www.jetbrains.com/phpstorm/}} \\ 

Zur Verwendung benötigt man allerdings eine Lizenz, die das Diplomarbeitsteam durch den Besuch an der HTL-Krems für ein Jahr kostenlos erhalten hat. 

Dadurch, dass das Projektteam bereits einige Erfahrungen mit PhpStorm bei schulischen Projekten gesammelt hat, war der Einsatz der Software bei dieser Projektarbeit im Vorhinein bereits klar. Die Entwicklung gestaltet sich durch die verschiedenen Hilfen, die die IDE bietet sehr einfach und intuitiv. Fehler, die in einfachen Texteditoren auftreten, wie zum Beispiel, dass Variablen noch nicht gesetzt wurden, oder sich deren Name bei der Verwendung unterscheidet, traten mithilfe PhpStorms bei der Diplomarbeit nicht auf, was einen immensen Zeitgewinn zur Folge hatte.


\subsection{Oracle-SQL-Developer}
Oracle-SQL-Developer ist eine grafische Version von SQL-Plus, die Datenbank-Entwicklern eine Möglichkeit gibt, grundlegende Aufgaben einfach zu erledigen. Diese können Datenbankobjekte betrachten, ansehen, bearbeiten und löschen. Des Weiteren kann man SQL-Abfragen an eine Datenbank senden. Außerdem besitzt die Software, die Möglichkeit Daten zu exportieren bzw. zu importieren ,wie zum Beispiel CSV- oder Excel-Dateien. Verbindungen zu einer Datenbank lassen sich mit einer Standardauthentifizierung (Benutzername, Passwort) aufbauen. Einmal verbunden, kann man Operationen in der Datenbank durchführen. (vgl. \cite{sqldeveloper})

Für detailliertere Informationen siehe Hersteller-Webseite.\footnote{\url{http://www.oracle.com/technetwork/developer-tools/sql-developer/overview/index-097090.html}} \\ 

Der Grund für den Einsatz in der Projektarbeit ist, dass die Software (nach einer Registrierung) frei erhältlich ist. Außerdem ist das Programm sehr einfach in der Handhabung und bietet einwn sehr großen Funktionsumfang, von der die Entwicklung sehr profitierte.



\subsection{XAMPP} 
Das Programm XAMPP stellt eine Sammlung verschiedenster freier Software dar. Es bietet eine einfache Installation und Konfiguration des Webservers Apache mit der relationalen Datenbank MariaDB und den Skript-Sprachen Perl und PHP. Das X im Namen der Software steht für die unterschiedlichen Betriebssystemen, auf denen es verfügbar ist (Linux, Microsoft Windows, Mac OS X, Solaris). Zusätzlich sind in der Sammlung noch andere nützliche Werkzeuge enthalten, wie ein FTP-Server, FileZilla, Mail-Server, phpMyAdmin, OpenSSL und Webalizer. Ziel des XAMPP-Projekts ist eine möglichst einfache Installation von Server-Werkzeugen zu bieten, die ansonsten lange konfiguriert werden müssten. (vgl. \cite{xampp})

Für detailliertere Informationen siehe Hersteller-Webseite.\footnote{\url{https://www.apachefriends.org/de/index.html}} \\ 

Zur Realisierung des Projekts wurde allerdings nur der Apache-Server und die Skript-Sprache PHP eingesetzt. XAMPP wurde verwendet, da es eine schnell einzurichtende lokale Entwicklungsumgebung bietet. Direkt nach der Installation kann mit der Entwicklung von Web-Applikationen begonnen werden. Der Apache-Server musste allerdings erweitert werden, damit er die Oracle-Datenbank abfragen und verarbeiten kann. Apache besitzt eine freie Software-Lizenz der Apache-Software-Foundation.


\subsection{Google Chrome}
Google Chrome ist ein Webbrowser des amerikanischen Unternehmens Google Inc. 
Das Projektteam hat den Browser hauptsächlich für die Entwicklung gewählt, da er so eine große Verbreitung findet und deshalb die Weboberfläche darauf angepasst werden muss. Auch bietet Google Chrome großartige Entwicklertools, die das Manipulieren des Quelltexts direkt auf der Webseite zulassen. Auch verfügt das Programm über eine Konsole, die über aufgetretene Fehler, wie zum Beispiel über eine fehlende CSS-Datei oder Fehler bei einer JavaScript-Methode informiert.

Für detailliertere Informationen siehe Hersteller-Webseite.\footnote{\url{https://www.google.de/chrome/browser/desktop/}} \\ 

\subsection{Mozilla Firefox}
Mozilla Firefox ist ein freier Webbrowser des Entwicklers Mozilla Corporation. 
Zur Entwicklung wurde ebenso Mozilla Firefox gewählt, da er von vielen Anwendern eingesetzt wird. Auch bietet er ähnliche hilfreiche Entwicklerwerkzeuge wie Google Chrome. Zudem kann man die Webseite mit einem nützlichen Tool untersuchen, dass den Aufbau der Seite in einer 3D-Ansicht offenlegt. Dies wird genutzt, um herauszufinden, ob vielleicht ein Element zu breit ist, wenn die Seite nicht wie gewünscht dargestellt wird.

Für detailliertere Informationen siehe Hersteller-Webseite.\footnote{\url{https://www.mozilla.org/de/firefox/new/}} \\ 

\subsection{Dropbox}
Dropbox ist eine freie Software, die Dateien in der Cloud (Speicherplatz im Internet) sichert. Wenn eine Datei auf Dropbox hochgeladen wurde, kann man sie von jedem internetfähigen Gerät abrufen. In der Standard-Version steht ausreichend Speicher zur Verfügung.

Das Projektteam hat Dropbox zum Dateiaustausch sowie zum Absichern der relevanten Dateien genutzt. Allerdings traten dabei auch einige Probleme auf, da des Öfteren Dateien in \glqq Konflikt\grqq{} standen, was bedeutet, dass beide Projektpartner die selbe Datei bearbeitet und gespeichert haben. Dies führte zu Dateien, die doppelt vorhanden waren. Dadurch mussten die Dateien analysiert werden, um die aktuelle zu finden. Auch sind einige Dateien bei der Entwicklung verloren gegangen, was mit Mehraufwand verbunden war, da diese erneut geschrieben werden mussten.

Für detailliertere Informationen siehe Hersteller-Webseite\footnote{\url{https://www.dropbox.com/de/}}. \\

\subsection{Texmaker}
Texmaker ist ein frei erhältlicher, moderner und auf vielen Plattformen erhältlicher Latex-Editor. Es beinhaltet viele Funktionalitäten (Rechtschreibprüfung, Autovervollständigung, usw.), die benötigt werden, um professionell Dokumente zu erstellen. (vgl. \cite{texmaker})

Aufgrunddessen, dass das Projektteam zuvor eine Einführung zu dieser Anwendung erhalten hat und daher bereits einige hilfreiche Funktionen bekannt waren, sowie durch die automatische Codevervollständigung wurde das Entwerfen des Dokuments um einiges beschleunigt.

Für detailliertere Informationen siehe Hersteller-Webseite.\footnote{\url{http://www.xm1math.net/texmaker/}} \\

