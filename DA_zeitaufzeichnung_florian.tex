\section{Stundenaufzeichnung}
In folgendem Abschnitt finden Sie die aufgewendeten Stunden der jeweiligen Projektpartner.
\subsection{Florian Molds Stunden}
\begin{longtable}[h]{| c | p{11.6cm} | l | l |}
\caption{Stundenaufzeichnung Florian}\label{tab:stundenaufzeichnung}\\ 
\toprule
* & Tätigkeit & Dauer (h) & Datum \\
\midrule
\endfirsthead
\caption[]{Stundenaufzeichnung Florian \small(Fortsetzung)}\\
\toprule
* & Tätigkeit & Dauer (h) & Datum \\
\midrule
\endhead
\midrule
\endfoot 
\bottomrule
\endlastfoot 
\hline

1 & 1. Besprechung mit Auftraggeber & 2 & 22.04.2015 \\ \hline

2 & 1. Besprechung mit Herrn Mestl & 1,25 & 25.04.2015 \\ \hline

3 & 2. Besprechung mit Auftraggeber & 1,5 & 29.05.2015 \\ \hline

4 & 2. Besprechung mit Herrn Mestl & 1 & 19.06.2015 \\ \hline

5 & Pflichtenheft schreiben & 6 & 28.08.2015 \\ \hline

6 & Pflichtenheft schreiben & 6 & 28.08.2015 \\ \hline

7 & Design mit Herrn Mestl besprochen & 2 & 20.09.2015 \\ \hline

8 & Einlesen in Oracle Datenbanktechnologie & 3 & 25.09.2015 \\ \hline

9 & Designentwicklung der Login-Seite in HTML & 2 & 02.10.2015 \\ \hline

10 & Designentwicklung der Lieferanten-Seite in HTML & 2,5 & 05.10.2015 \\ \hline

11 & EWA: Einführung in wissenschaftliches Arbeiten & 3 & 05.10.2015 \\ \hline

12 & Designentwicklung der Buchhalter-Seite in HTML & 3 & 10.10.2015 \\ \hline

13 & ER-Modell Entwurf der Datenbank & 1,5 & 15.10.2015 \\ \hline

14 & 3. Besprechung mit Herrn Mestl & 0,4 & 16.10.2015 \\ \hline

15 & Auswahl des geeigneten Entwicklerframeworks für das Projekt & 2 &  20.10.2015 \\ \hline

16 & Installation des Laravel-Frameworks & 1 &  30.10.2015 \\ \hline

17 & Einarbeiten in das Laravel-Framework & 3 & 02.11.2015 \\ \hline

18 & EWA: Einführung in wissenschaftliches Arbeiten & 3 & 03.11.2015 \\ \hline

19 & Bereits vorhandene Seiten in die Laravel Template Engine integrieren. & 4 & 27.11.2015 \\ \hline

20 & EWA: Einführung in wissenschaftliches Arbeiten & 3 & 10.11.2015 \\ \hline

21 & 4. Besprechung mit Herrn Mestl & 0,3 & 20.11.2015 \\ \hline

22 & EWA: Einführung in wissenschaftliches Arbeiten & 4 & 15.12.2015 \\ \hline

23 & Installation des Oracle Express Edition Datenbanksystems & 2 & 12.01.2016 \\ \hline

24 & Login-Funktionalität für Lieferanten umsetzen & 3 & 20.01.2016 \\ \hline

25 & 5. Besprechung mit Herrn Mestl & 0,5 & 29.01.2015 \\ \hline

26 & Registrierung für Lieferanten realisiert & 3 & 29.01.2016 \\ \hline

27 & Verfassen eines groben Inhaltsverzeichnisse für die Diplomarbeit & 2 &  29.01.2016 \\ \hline

28 & Registrierung fertiggestellt & 1 & 30.01.2016 \\ \hline

29 & Implementierung einer Passwort-Vergessen Form & 2 & 31.01.2016 \\ \hline

30 & Passwortkriterien müssen beim Regisitrieren beachtet werden & 2 & 01.02.2016 \\ \hline

31 & Passwort-Vergessen Mail-Benachrichtigung funktioniert &  3 & 02.02.2016 \\ \hline

32 & Beim Login wird für jede Benutzergruppe eine Session erstellt & 1 & 03.02.2016 \\ \hline

33 & Rechnungen werden aus der Datenbank ausgelesen und in die Tabellen geschrieben & 2 & 03.02.2016 \\ \hline

34 & Login-Fehler behoben + Erfolgs- / Fehlermeldungen & 1 & 03.02.2016 \\ \hline

35 & Passwort-Ändern Buchhalter, Lieferant und Administrator & 3 & 05.02.2016 \\ \hline

36 & Passwort-Ändern Fehlerbehebung & 0,5 & 08.02.2016 \\ \hline

37 & 404 Fehlerseite erstellt & 1 & 08.02.2016 \\ \hline

38 & Prüfen ob z.B.: Benutzer die Lieferantenseite betrachten darf & 1 & 08.02.2016 \\ \hline

39 & Rechnung löschen & 1 & 08.02.2016 \\ \hline

40 & Rechnung im Browser betrachten & 0,5 & 08.02.2016 \\ \hline

41 & Abfrage ob die Rechnung wirklich gelöscht werden soll & 0,5  & 09.02.2016 \\ \hline

42 & Metadaten für jede einzelne Rechnung ausgelesen um im zugehörigen Modal angezeigt & 1 & 09.02.2016\\ \hline

43 & Benachrichtigung für jede Rechnung ausgelesen & 0,5 & 09.02.2016 \\ \hline

44 & Logging Funktionaltität begonnen & 1 & 09.02.2016 \\ \hline

45 & Design-Verbesserung der Backend-Seite & 1 & 09.02.2016 \\ \hline

46 & Weitere Verbesserung der Backend Seite & 2 & 10.02.2016 \\ \hline

47 & Mail wird bereits mit Rechnung als Anhang versendet & 1 & 10.02.2016 \\ \hline

48 & Zusätzlich zur Benachrichtigungsspeicherung in der Datenbank wird die Benachrichtigung als Mail an die Buchhaltung versendet. & 1,5 & 11.02.2016 \\ \hline

49 & Das Intervall, wann das Passwort geändert werden muss, greift beim Login vom Administrator, Buchhalter und Lieferant & 2,5 & 12.02.2016 \\ \hline

50 & XML-Datei lässt sich bereits generieren, allerdings fehlen noch die richtigen Elemente & 1 & 12.02.2016 \\ \hline

51 & Die Mails die absgesendet werden müssen, werden nicht sofort verschickt, sondern in eine Warteschlange geschoben und später versendet. Dies verbessert die Response-Zeit der Anwendung, da die Benutzer nicht warten müssen, bis die Mail verschickt ist, sondern sie können gleich weiterarbeiten & 2 & 12.02.2016 \\ \hline

52 & Direkt nach dem Versenden kann der Lieferant, seine Benachrichtigungen betrachten. & 0,5 & 12.02.2016 \\ \hline

53 & 403 Fehlerseite erstellt & 0,5  & 12.02.2016 \\ \hline

54 & Code-Dokumentation der Loginklasse, Buchhaltungsklasse, Lieferantenklasse, Methodsklasse zum selbst erstellten Programmcode & 2 & 12.02.2016 \\ \hline

55 & Die Registrierform wird nach der Registrierung geleert, vorher blieben alle Werte darin enthalten & 0,5 & 13.02.2016 \\ \hline

56 & Code-Dokumentation der HTML-Seiten, Datenbanktabellen, Migrations & 3 & 13.02.2016 \\ \hline

57 & Fehlerbehebung bei der Lieferantenansicht. & 0,25 & 13.02.2016 \\ \hline

58 & Anpassung der Login-Seite für mobile Desktops & 1,5 & 13.02.2016 \\ \hline

59 & Anpassung der Buchhaltungs- / Lieferantenseite für mobile Desktops & 1 & 14.02.2016 \\ \hline

60 & Anpassung der Passwort-Vergessen und Passwort-Ändern Seite für mobile Desktops & 0,5 & 14.02.2016 \\ \hline

61 & Jeden Tag um 24 Uhr werden bereits geholte Rechnungen + xmls gelöscht & 1 & 16.02.2016 \\ \hline

62 & Sicherung des Standes der kompletten Datenbank wird jeden Tag um 24 Uhr durchgeführt & 1 & 16.02.2016 \\ \hline

63 & Tabellen der Buchhaltungsseite / Lieferantenseite und Administratorseite für mobile Auflösungen optimiert & 1 & 17.02.2016 \\ \hline

64 & Ablaufdiagramm für den Login erstellt & 1 & 18.02.2016 \\ \hline

65 & Ablaufdiagramm für die Registrierung erstellt & 1 & 18.02.2016 \\ \hline

66 & Ablaufdiagramm für das Rechnung holen erstellt & 1 & 18.02.2016 \\ \hline

67 & Ablaufdiagramm für den Passwort-Vergessen erstellt & 1 & 18.02.2016 \\ \hline

68 & EWA: Einführung in wissenschaftliches Arbeiten & 3 & 23.02.2016 \\ \hline

69 & 3. Besprechung mit Auftraggeber & 2 & 24.02.2016 \\ \hline

70 & 6. Besprechung mit Herrn Mestl & 1 & 26.02.2016 \\ \hline

71 & XML-Datei mit den richtigen Elementen, Attributen und Werten wird erstellt & 1,5 & 24.02.2016 \\ \hline

72 & Logging Funktion für alle geforderten Funktionen implementiert & 2 & 25.02.2016 \\ \hline

73 & Erweiterung der Datenbank um einen Rechnungsstatus wie z.B.: gelöscht, geholt, ... & 0,5 & 25.02.2016 \\ \hline

74 & Kleine Fehlerbehebung beim Benachrichtigung anzeigen / senden & 0,5 & 25.02.2016 \\ \hline

75 & Alle offenen Rechnungen ohne Benachrichtigung lassen sich von der Buchhaltung gleichzeitig holen & 1,5 & 27.02.2016 \\ \hline

76 & Überprüfung, ob eine zu holende Rechnung bereits von einem anderen Buchhaltungsbenutzer geholt wurde & 1 & 27.02.2016 \\ \hline

77 & Fehlerbehebung bei den Datatables. Diese werden nun automatisch vergrößert/verkleinert wenn sich die Auflösung der Seite ändert. Nachdem die Seite aktualisiert wurde, kommt man wieder auf den richtigen Tab beim Tabcontrol, wo man zuvor war. & 2,5 & 28.02.2016 \\ \hline

78 & Rechnung können nun nicht mehr von allen Benutzern gelesen werden, sondern nur noch von authentifizierten Anwendern. & 2 & 28.02.2016 \\ \hline

79 & Verfassen der Einleitung der Diplomarbeit & 2 & 29.02.2016 \\ \hline

80 & Schreiben des ersten Kapitels & 4 & 29.02.2016 \\ \hline

81 & Umsetzung des aktuellen Standes der Stunden in das Dokument & 1,5 & 29.02.2016 \\ \hline

82 & Beginn des Verfassens von Testfällen & 2 & 29.02.2016 \\ \hline

83 & Vervollständigen der Testfälle & 2 & 31.02.2016 \\ \hline

84 & Beschreiben der Funktionen (Registrieren, Login) & 3 & 01.03.2016 \\ \hline

85 & Beschreiben der Funktionen (Rechnung holen, Passwort vergessen & 2 & 02.03.2016 \\ \hline

86 & Beschreiben von Laravel & 3 & 04.03.2016 \\ \hline

87 & Beschreiben von Composer, PHP, XML & 3 & 05.03.2016 \\ \hline

88 & Beschreiben von Latex, Phpstorm & 2 & 06.03.2016 \\ \hline

89 & Beschreiben von SQL Developer, XAMPP & 2 & 08.03.2016 \\ \hline

90 & Beschreiben von Chrome, Firefox & 1 & 10.03.2016 \\ \hline

91 & Beschreiben von Dropbox, Texmaker & 2 & 13.03.2016 \\ \hline

92 & Einfügen eines Bildes des Datenbankmodells + Beschreibung dieser & 3 & 14.03.2016 \\ \hline

93 & Einfügen einer Gender-Erklärung in das Dokument & 0,5 & 15.03.2016 \\ \hline

94 & Übersetzen der Einleitung ins Englische & 1,5 & 18.03.2016 \\ \hline

95 & Testfälle von Michael getestet & 4 & 19.03.2016 \\ \hline

96 & Testfälle allgemein verfasst & 0,5 & 20.03.2016 \\ \hline

97 & Einrichtung der Anwendung bei ELK Fertighaus GmbH & 2,5 & 21.03.2016 \\ \hline

98 & E-Mail wird auch an Administrator versendet, wenn sich ein Lieferant registriert & 2,5 & 26.03.2016 \\ \hline

\end{longtable}
